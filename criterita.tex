\documentclass[10pt,a4paper]{article}
\usepackage[utf8]{inputenc}
\usepackage{amsmath}
\usepackage{amsfonts}
\usepackage{amssymb}
\usepackage{hyperref}
\usepackage{geometry}
\usepackage{listliketab}
\usepackage{array}
\usepackage{longtable}

\usepackage{natbib}
\usepackage{bibentry}
\begin{document}

\section{Essential Criteria}
\subsection{Proven excellent programming skills, including iOS development, JavaScript (client and server-side), ontologies, and rules.}
I have completed freelance iOS projects for the iPad that utilise multiple features of the iOS SDK and I have been using client-side javascript since my undergraduate, to perform AJAX requests and using the JQuery library to provide smooth animations in single page sites. When Node.JS began to increase in popularity, I learnt the basics in a few days and co-presented an introduction to colleagues, as well as developing Node.JS applications during hackathons and deploying them to the cloud service, Heroku. 

My PhD has been focussed on the role of knowledge in a wireless sensor network to make more efficient use of the limited resource, and to deliver more valuable data faster. During the course of this, I have developed a linking ontology that links (and extends) three ontologies within the scientific observation and sensor hardware space. As far as we can tell, this is the first ontology of its kind and we are working on making it applicable to all sensor networks that deal with scientific observations.

I have experience with the Droos rule engine, using it to encode knowledge into rules, and firing these rules over data that has originated from sensors. Using the rule engine as a backend, I have also developed an API that allows POST requests to be issued in order to manipulate the knowledge base, as well as the data it fires upon.

\subsection{Proven excellent oral and interpersonal skills, including working with end-users (non-computer scientists).}
The field work of my PhD was based in Malaysia, and involved me working with, and interviewing, ecologists and biologists to ensure that the system satisifes all the requirements of someone working within the field. This collaboration also involved presenting and explaining my work to those without any prior knowledge of computer science.

\subsection{Proven excellent written communication skills.}
I have two peer-reviewed papers published and I have provided regular progress updates throughout the PhD to both improve my written skills, and show my progress in a clear, concise manner that can be reused when writing the thesis. During my field work, which involved two weeks in Malaysia each year, I wrote blog posts every day that, while providing technical details, catered for all audiences.

\subsection{Proven good organisational skills.}
For the last two and a half years, I have been co-organising a weekly, student-led seminar series for postgradate students within the School (FTS). This has meant refreshments, room bookings and speakers must be organised for each semester, as well as providing introductory talks on new technologies/PhD tips whenever students request them. Due to the practical nature of the PhD, I have had to organise yearly trips, that involve the particiaption of multiple schools, and conference attendance, as well as ensuring that papers written are submitted on time.

\subsection{Sufficient breadth or depth of specialist knowledge in the discipline to develop further skills in and knowledge of semantic sensing systems.}
My PhD project is focussed on developing a sensing system that is able to use the knowledge of its surrounding environment, as well as knowledge from data sensed. I have published and presented on the subject and I ensure that I am kept up to date through literature reviews, alerts from Google Scholar and attending conferences.

\subsection{Knowledge of good software engineering practice in relation to mobile "app" development, Web APIs, and sensor information processing systems.}
Experience of developing software professionally (freelance and otherwise) has assisted my knowledge of software engineering practice, as well as supervising undergraduate group projects to follow the Agile development methodology. 

Mobile app development, both professionally and in my own time, requires strict project management and the use of version control (such as git), as well as rigorous testing to ensure the application works on all target devices. In order to keep up with the fast pace of mobile development, I follow trends, common practices and watch any developer keynotes of relevance.

\subsection{Experience of working with sensor middleware and sensor data processing libraries.}
Working with the Global Sensor Networks (GSN) Java middleware for sensors has provided me with a great deal of knowledge on the components required to build a robust and extensible middleware. However, dealing with data on devices with differing processing capabilities has also provided the experience of processing raw sensed data on devices with limited capabilities, as well as full desktop machines.

Field work has allowed me to work with various state-of-the-art sensing devices, that all use different languages and libraries, to process real world data and run it through our middleware.

\subsection{Self-motivated with a proactive approach to work.}
The individual nature of the PhD has encouraged me to find working practices that increase my productivity. I have also been actively involved in the organisation of three succesful hackathons, some of which have been interdisciplinary.

Not all new technologies have been applicable to my PhD and, as can be seen in my \href{http://github.com/encima/}{Github}, I have many side-projects that I use to gain skills outside of the PhD and I am actively involved in Cardiff's developer community; going to events like  \href{http://unifieddiff.co.uk}{Unified Diff} \\

\subsection{Ability to work towards identified goals within an agreed timeframe.}
The requirement of the PhD program is to provide a report every six months on my progress and no deadlines have been missed. Working in freelance, I have delivered every milestone ahead of the deadline and learnt to adjust timeframes accordingly as soon as the specification of a product changes. When I have submitted papers, all of my work has been submitted before the deadline and revisions have been provided in a timely manner.

\subsection{Qualifications: A good first degree (1st /2:1) in Computer Science or a related discipline, or equivalent experience.}
I achieved a 2:1 BSc(Hons) in Computer Science from Cardiff University.

\section{Desirable Criteria}
\subsection{Experience in working with image processing or natural language processing packages.}
I have experience with both OpenCV (a C++ based tool for image and video manipulation), as well as SimpleCV (a Python based tool that provides similar functionality). Using these tools has allowed me to build a background model for a small set of images and remove it, extracting the object of interest from the foreground (if any). 
Using tools like Weka (a data mining tool), I have performed some natural language processing on metadata extracted from thousands of images, as well as on interviews of researchers in the field.

\subsection{Experience in working in an international team.}
Due to working in an inter-disciplinary team for my PhD, I have experience of working with people both inside and out of the European Union. Much of my field work was based in Malaysia and involved working with academics there, as well as in Cardiff. This forced meetings to be made well in advance, as well as allowances for remote meetings and regular email contact.

\subsection{Ability to present confidently to academic and non-academic audiences at meetings and conferences.}
I have presented at two conference, one in the Mobile Systems space and the other in Biodiversity. During my work in Malaysia, I have presented to visitors, to the field centre where I was based, that were not academics.

Presentations to peers, whether it is through student-led seminars or yearly reviews, have proved to be a useful way of getting feedback on your presentation skills, as well as your work.

\end{document}