
\documentclass[11pt,fullpage]{article}

\usepackage{hyperref}
\usepackage{geometry}
\usepackage{listliketab}
\usepackage{array}
\usepackage{longtable}

\usepackage{natbib}
\usepackage{bibentry}

\nobibliography*
\bibliographystyle{plain}

% Palatino font
%\usepackage[T1]{fontenc}
%\usepackage[sc,osf]{mathpazo}

\def\name{Christopher Gwilliams}

\reversemarginpar

\geometry{
  body={6.5in, 8.5in},
  left=1.0in,
  top=1.0in
}

% Customize page headers
\pagestyle{myheadings}
\markright{\name}
\thispagestyle{empty}

% Custom section fonts
\usepackage{sectsty}
\sectionfont{\rmfamily\mdseries\Large}
\subsectionfont{\rmfamily\mdseries\itshape\large}

% Other possible font commands include:
% \ttfamily for teletype,
% \sffamily for sans serif,
% \bfseries for bold,
% \scshape for small caps,
% \normalsize, \large, \Large, \LARGE sizes.

% Don't indent paragraphs.
\setlength\parindent{0em}

% Make lists without bullets
\renewenvironment{itemize}{
  \begin{list}{}{
    \setlength{\leftmargin}{1.5em}
  }
}{
  \end{list}
}

\begin{document}
\bibliographystyle{apalike}

% Print name centered and bold:
\centerline{\Large \bf \name}

\vspace{0.25in}

\begin{minipage}{0.50\linewidth}
  \href{http://cf.ac.uk/}{Cardiff University} \\
  \href{http://cs.cf.ac.uk}{School of Computer Science and Informatics} \\
  Queens Buildings \\
Cardiff, CF24 3AA
\end{minipage}
\begin{minipage}{0.50\linewidth}
  \begin{tabular}{ll}
    Phone: & 07725117838 \\
    Email: & \href{mailto:scm7cg@cs.cf.ac.uk}{scm7cg@cs.cf.ac.uk} \\
    Homepage: & \href{http://www.encima.co.uk/}{www.encima.co.uk} \\
    Github: & \href{http:/github.com/encima/}{www.github.com/encima} \\
  \end{tabular}
\end{minipage}

\section*{Research Interests}
My current research interests focus around the use of new technologies to develop `intelligent' systems that are capable of using knowledge, stored in a predefined format, to enrich the data that passes through them.

\section*{Publications}

\setlength{\extrarowheight}{10pt}

\begin{longtable}{p{0.5in}|p{5.5in}}
  2012 & \bibentry{gwilliams2012local} \\
  	   & \bibentry{gwilliams2012k} \\
\end{longtable}

\section*{Posters}

\begin{longtable}{p{0.5in}|p{5.5in}}
  2012 & \bibentry{gwilliams2012poster} \\
\end{longtable}

%\setlength{\extrarowheight}{5pt}
\newpage
\section*{Education}
\subsection*{\textbf{2010-Present: PhD} in Computer Science, Cardiff University}
\subsubsection*{Using Local and Global Knowledge in Wireless Sensor Networks
 (Funded by the EPSRC)}
Supervisors: Prof. Alun Preece and Mr. Alex Hardisty
\subsubsection*{Research Synopsis}
My work focusees on increasing the efficiency of wireless sensor networks (WSNs) by using local and global knowledge. We believe that this knowledge can be encoded into WSNs in order to make informed routing decisions on the data collected. More importantly, through using sensing devices with increased knowledge-processing capabilities, the profit is increased. We define profit is the value of the sensed data received by the end user.

Throughout the PhD, I have used a variety of technologies, such as:
\begin{enumerate}
	\item Drools - A Java-based rule management system with a forward chaining inference based rule engine. This is used at each sensor in the WSN in order to process sensed data based on the knowledge stored within its knowledge base. As new data is received, the knowledge base is dynamically updated based on the contents of the data.
	\item Global Sensor Networks (GSN) - A Java-based sensor middleware that allows the description of sensors through XML files and provides a web-based front end to view the sensed data. We use GSN to add each sensing device and to control the sending and receiving of sensed data within the network.
	\item Protege - A Java-based desktop application for developing OWL ontologies. I have used this to create an ontology to link and extend existing ontologies for scientific observations.
	\item OpenCV - A C++ computer vision framework for the manipulation of images and video. I have used this in order to remove busy backgrounds from images and extract the largest object in the foreground.
	\item Node.JS - A Javascript platform used to build network applications. Node.JS provides an interface to the data held by GSN, as well as an interface to control the Drools rule engine and add new data into the network. I have also implemented APIs in Node that direct manipulation of the rule engine.
\end{enumerate}
	
\subsection*{\textbf{2007-2010: BSc(Hons)} in Computer Science, Cardiff University}
\subsubsection*{\textbf{Individual Project:} An Automated Helpdesk Application for the Resolution of Hardware and Software Faults in an I.T. Support Company}

\section*{Administrative Experience}

\begin{tabular}{>{\everypar{\hangindent0.5in}}p{6in}}
	\textbf{PhD Student Representative}: During the course of my PhD, I have held the position as the postgraduate representative for Computer Science for three years. This has involved taking feedback from students, attending student/staff panels and providing action plans for any issues that may arise in the postgraduate offices. \\
	\textbf{University Graduate College (UGC) Representative}: Also throughout the PhD, I have been one of the Physcial Sciences representatives for the UGC, sitting on the board for `Traning and Development'. During this period I have assisted with the implementation of professional teaching recognition for PhD students within Cardiff University and the streamlining of courses offered by the University to maximise cost effectiveness. \\
\end{tabular}

\section*{Teaching Experience}

\begin{tabular}{>{\everypar{\hangindent0.5in}}p{6in}}
	2010-2013: \textbf{Teaching Assistant}. I have taught a range of courses for both undergraduates and Masters students, including, but not limited to: Web development and APIs, Relational and Object-Relational Databases, Mobile App Development and Software Development Practises. Recognised as an Associate Fellow of the HEA.\\
	2010-2013: \textbf{Group Supervisor}. I supervised multiple second year group projects that allow students to follow a `real world' project through from the requirements gathering stage to implementation.  This involved: marking reports, ongoing supervision, knowledge of software development processes and the latest technology (should they choose to use new tools).\\
\end{tabular}

\section*{Professional Experience}

\begin{tabular}{>{\everypar{\hangindent0.5in}}p{6in}}
	2011: \textbf{iOS and Android Development}. Worked with Dr. Ian Grimstead to create an application to encourage engagement in areas (i.e. Hospitals) where members of the public 		can text their views and have them displayed on a large monitor. Involved an Android app to receive and forward the texts with an iPad app to receive SMS and design screens to be 			output by the device. \\
	\textbf{Other Android Development}: Since 2010, I have developed and published 4 Android apps to the Google Play Store. The majority have received over 1,000 downloads and ratings of higher than 	4 (out of 5). \\
	\textbf{Web Development}: Co-organising a postgraduate seminar series each week has given me the opportunity to learn new technologies in a short time and present it to colleagues. During this 	time, I have presented introductions to Django, Flask, Node.JS and others. I have also developed sites with these tools at hackathons (an example can be seen at 						    \href{http://eartub.es}{eartub.es} and have also undertaken freelance web development.
\end{tabular}

\section*{Professional Membership}

\begin{tabular}{>{\everypar{\hangindent0.5in}}p{6in}}
	Higher Education Academy (HEA) : Associate Fellow (2012-Present) \\
	Science, Technology, Engineering and Maths (STEMNET): Ambassador (2010-Present)
\end{tabular}
% Footer
\bigskip
\begin{center}
  \begin{footnotesize}
    Last updated: \today
  \end{footnotesize}
\end{center}

\nobibliography{citations}
\end{document}